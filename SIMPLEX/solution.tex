\documentclass{article}\usepackage{amsmath}\begin{document}\title{Simplex Solver}\maketitle\begin{flushleft}\textbf{Problem}\end{flushleft}\begin{flushleft}Dado o seguinte sistema linear e objetivofunção, encontre a solução ideal.\end{flushleft}\begin{equação*}\max{ 5x_1 + 5x_2 + 5x_3 } \\ \end{equation*}\[\left\{\begin{array}{c} 6x_1 + 3x_2 + 5x_3 \leq 500 \\  5x_1 + 4x_2 + 3x_3 \leq 350 \\  3x_1 + 2x_2 + 2x_3 \leq 150 \\  x_3 \leq 20 \\ \end{array}\right.\]\begin{flushleft}\textbf{Solution}\end{flushleft}\begin{flushleft}Adicione variáveis ​​de folga para girartodas as desigualdades em igualdades.\end{flushleft}\[\left\{\begin{array}{c} 6x_1 + 3x_2 + 5x_3 + s_1 = 500 \\  5x_1 + 4x_2 + 3x_3 + s_2 = 350 \\  3x_1 + 2x_2 + 2x_3 + s_3 = 150 \\  x_3 + s_4 = 20 \\ \end{array}\right.\]\begin{flushleft}Crie o quadro inicial do novo sistema linear.\end{flushleft}\begin{equation*}\begin{bmatrix}\begin{array}{ccccccc|c}x_1 &x_2 &x_3 &s_1 &s_2 &s_3 &s_4 &b \\ \hline6 & 3 & 5 & 1 & 0 & 0 & 0 & 500 \\5 & 4 & 3 & 0 & 1 & 0 & 0 & 350 \\3 & 2 & 2 & 0 & 0 & 1 & 0 & 150 \\0 & 0 & 1 & 0 & 0 & 0 & 1 & 20 \\ \hline-5 & -5 & -5 & 0 & 0 & 0 & 0 & 0 \\\end{array}\end{bmatrix}\begin{array}{c}\\s_1 \\s_2 \\s_3 \\s_4 \\\\\end{array}\end{equation*}\begin{flushleft}Existem elementos negativos na linha inferior,então a solução atual não é a ideal.Assim, gire para melhorar a solução atual. O a variável de entrada é $x_1$ e a variável de saída variável é $s_3$.\end{flushleft}\begin{flushleft}Execute operações elementares de linha até o elemento pivô é 1 e todos os outros elementos no coluna de entrada é 0.\end{flushleft}\begin{equation*}\begin{bmatrix}\begin{array}{ccccccc|c}x_1 &x_2 &x_3 &s_1 &s_2 &s_3 &s_4 &b \\ \hline0 & -1 & 1 & 1 & 0 & -2 & 0 & 200 \\0 & 2/3 & -1/3 & 0 & 1 & -5/3 & 0 & 100 \\1 & 2/3 & 2/3 & 0 & 0 & 1/3 & 0 & 50 \\0 & 0 & 1 & 0 & 0 & 0 & 1 & 20 \\ \hline0 & -5/3 & -5/3 & 0 & 0 & 5/3 & 0 & 250 \\\end{array}\end{bmatrix}\begin{array}{c}\\s_1 \\s_2 \\x_1 \\s_4 \\\\\end{array}\end{equation*}\begin{flushleft}Existem elementos negativos na linha inferior,então a solução atual não é a ideal.Assim, gire para melhorar a solução atual. O a variável de entrada é $x_2$ e a variável de saída variável é $x_1$.\end{flushleft}\begin{flushleft}Execute operações elementares de linha até o elemento pivô é 1 e todos os outros elementos no coluna de entrada é 0.\end{flushleft}\begin{equation*}\begin{bmatrix}\begin{array}{ccccccc|c}x_1 &x_2 &x_3 &s_1 &s_2 &s_3 &s_4 &b \\ \hline3/2 & 0 & 2 & 1 & 0 & -3/2 & 0 & 275 \\-1 & 0 & -1 & 0 & 1 & -2 & 0 & 50 \\3/2 & 1 & 1 & 0 & 0 & 1/2 & 0 & 75 \\0 & 0 & 1 & 0 & 0 & 0 & 1 & 20 \\ \hline5/2 & 0 & 0 & 0 & 0 & 5/2 & 0 & 375 \\\end{array}\end{bmatrix}\begin{array}{c}\\s_1 \\s_2 \\x_2 \\s_4 \\\\\end{array}\end{equation*}\begin{flushleft}Não há elementos negativos na linha inferior, então sabemos que a solução é ótima. Assim, a solução é: \end{flushleft}\begin{equation*}s_1 = 275, s_2 = 50, s_3 = 0, s_4 = 20, x_1 = 0, x_2 = 75, x_3 = 0, z = 375\end{equation*}\end{document}